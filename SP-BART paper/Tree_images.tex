\documentclass[APA,LATO1COL]{WileyNJD-v2}
\usepackage[linguistics]{forest, adjustbox}
\usepackage{dsfont}
\usepackage{graphicx}
\usepackage{subcaption}
\usepackage{algorithm, setspace}
\usepackage{algpseudocode} 
\usepackage{mwe}
\articletype{Article Type}%

\raggedbottom

\begin{document}
  \begin{figure*}
  \captionsetup[subfigure]{labelformat=empty}
        \centering
        \begin{subfigure}[b]{0.475\textwidth}
            \centering
            \caption[Network2]%
            {{ \footnotesize $\mathcal{T}^{(1)}_{1}$}}
            \begin{forest}
for tree={
    grow=south, draw, minimum size=3ex, 
    inner sep=3pt, % control the overall tree size
    s sep=7mm,
    l sep=6mm
    }
[$ \mbox{ } x_{2} < 1 \mbox{ } $,
    [$ \mbox{ } x_{1} < 0.5 \mbox{ } $, edge label={node[midway, font=\footnotesize, left]{$\mbox{ False }$}}
        [$\mu_{11}$, circle,  edge label={node[midway,left, font=\footnotesize]{$\mbox{ False }$}},]
        [$\mu_{12}$, circle,  edge label={node[midway,right, font=\footnotesize]{$\mbox{ True }$}},]
    ]
    [$\mu_{13}$, circle,  edge label={node[midway,right, font=\footnotesize]{$\mbox{ True }$}},]
]
\end{forest}
            \label{fig:mean and std of net14}
        \end{subfigure}
        \hfill
        \begin{subfigure}[b]{0.475\textwidth}  
            \centering 
            \caption[]%
            {{\footnotesize $\mathcal{T}^{(2)}_{1}$}}
\begin{forest}
for tree={
    grow=south, draw, minimum size=3ex, 
    inner sep=3pt, % control the overall tree size
    s sep=7mm,
    l sep=6mm
        }
[$ \mbox{ } x_{2} < 1 \mbox{ } $,
    [$ \mbox{ } x_{1} < 0.5 \mbox{ } $,
        [$\mu_{11}$, circle]
        [$\mu_{12}$, circle]
    ]
        [$ \mbox{ } x_{4} < 0.75 \mbox{ } $,
        [$\mu_{13}$, circle]
        [$\mu_{14}$, circle]
    ]
]
\end{forest}
            \label{fig:mean and std of net24}
        \end{subfigure}
        \vskip\baselineskip
        \begin{subfigure}[b]{0.475\textwidth}   
            \centering 
                        \caption[]%
            {{\footnotesize $\mathcal{T}^{(3)}_{1}$}}    
\begin{forest}
for tree={
    grow=south, draw, minimum size=3ex, inner sep=3pt, s sep=7mm, l sep=6mm
        }
[$ \mbox{ } x_{2} < 1 \mbox{ } $,
    [$ \mbox{ } x_{1} < 0.5 \mbox{ } $,
        [$\mu_{11}$, circle]
        [$\mu_{12}$, circle]
    ]
    [$ \mbox{ } x_{3} < 2 \mbox{ } $,
        [$\mu_{13}$, circle]
        [$\mu_{14}$, circle]
    ]
]
\end{forest}
            \label{fig:mean and std of net34}
        \end{subfigure}
        \hfill
        \begin{subfigure}[b]{0.475\textwidth}   
            \centering 
            \caption[]%
{{\footnotesize $\mathcal{T}^{(4)}_{1}$}}  
            \begin{forest}
for tree={
    grow=south, draw, minimum size=3ex, inner sep=3pt, s sep=7mm, l sep=6mm
        }
[$ \mbox{ } x_{2} < 1 \mbox{ } $,
    [$ \mbox{ } x_{3} < 2 \mbox{ } $,
        [$\mu_{11}$, circle]
        [$\mu_{12}$, circle]
    ]
    [$ \mbox{ } x_{1} < 0.5 \mbox{ } $,
        [$\mu_{13}$, circle]
        [$\mu_{14}$, circle]
    ]
]
\end{forest}
\end{subfigure}
\caption[ ]
        {\normalsize An example of a tree generated from BART in 4 different instances. In principle, BART does not generate only one tree but rather a set of trees that summed together are responsible for the final prediction. The tree is represented as $\mathcal{T}^{(r)}_{1}$, where $(r) = 1,2,3,4$ denotes the number of the iteration in which the tree is updated. The splitting rules (covariates and their split points) are presented in the internal nodes (squares). The predicted values $\mu_{t \ell}$ are shown inside the terminal nodes (circles). $\mathcal{T}^{(1)}_{1}$ illustrates the tree at iteration one with three terminal nodes (circles) and two internal nodes (squares). From $\mathcal{T}^{(1)}_{1}$ to $\mathcal{T}^{(2)}_{1}$, the growing move is illustrated, as $\mu_{13}$ in $\mathcal{T}^{(1)}_{1}$ is split into $\mu_{13}$ and $\mu_{14}$ in $\mathcal{T}^{(2)}_{1}$ by using $x_{4} < 0.75$. In addition, the pruning process can be seen when $\mathcal{T}^{(2)}_{1}$ reverts to $\mathcal{T}^{(1)}_{1}$. The change move is shown when comparing $\mathcal{T}^{(2)}_{1}$ and $\mathcal{T}^{(3)}_{1}$, as the splitting rule that defines $\mu_{13}$ and $\mu_{14}$ is changed from $x_{4} < 0.75$ to $x_{3} < 2$. Finally, the swap movement is illustrated in the comparison of $\mathcal{T}^{(3)}_{1}$ and $\mathcal{T}^{(4)}_{1}$.} 
        \label{BCART}
    \end{figure*}



\begin{figure}[!htb]
\centering
            \begin{forest}
for tree={
    grow=south, draw, minimum size=3ex, inner sep=3pt, s sep=7mm, l sep=6mm
        }
[$ \mbox{ } x_{1} < 0.5 \mbox{ } $,
    [$ \mbox{ } x_{2} < 0.5 \mbox{ } $,
        [$\mbox{ } 4 \mbox{ }$, circle]
        [-7, circle]
    ]
    [$ \mbox{ } x_{2} < 0.5 \mbox{ } $,
        [$\mbox{ } 3 \mbox{ }$, circle]
        [-8, circle]
    ]
]
\end{forest}
\caption{An example of a tree generated from $\mathcal{T}_{1}$. In if-else format this can be written as $\mathcal{T}_{1} = 4\mathds{1}(x_{i1} < 0.5) \times \mathds{1}(x_{i2} < 0.5) -7 \mathds{1}(x_{i1} < 0.5)\times \mathds{1}(x_{i2} \geq 0.5) + 3 \mathds{1}(x_{i1} \geq 0.5)\times \mathds{1}(x_{i2} < 0.5) - 8 \mathds{1}(x_{i1} \geq 0.5)\times \mathds{1}(x_{i2} \geq 0.5)$, where $\mathds{1}(\cdot)$ denotes the indicator function.}
\label{LM_BART_tree}
\end{figure}
\end{document}